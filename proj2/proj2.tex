\documentclass[a4paper, 11pt, twocolumn]{article}
\usepackage[top=2.3cm, left=1.4cm, text={18.2cm, 25.2cm}]{geometry}
\usepackage[czech]{babel}
\usepackage[utf8]{inputenc}
\usepackage[IL2]{fontenc}
\usepackage[hidelinks, linkcolor=black, unicode]{hyperref}
\usepackage{bookmark}
\usepackage{nameref}
\usepackage{times}
\usepackage{amsthm} 
\usepackage{amsmath}
\usepackage{amsfonts}

\begin{document}

\newtheorem{definice}{Definice}
\newtheorem{veta}{Věta}

\thispagestyle{empty}
\begin{titlepage}
    \begin{center}
        \textsc{\Huge Vysoké učení technické v~Brně\\[0.5em]
        \huge Fakulta informačních technologií}\\
        \vspace{\stretch{0.382}}
        {\LARGE Typografie a publikování\,--\,2.\ projekt \\[0.4em]}
        {\LARGE Sazba dokumentů a matematických výrazů}
        \vspace{\stretch{0.618}}
    \end{center}
    {\Large 2023 \hfill Marek Gergel (xgerge01)}
\end{titlepage}

\section*{Úvod}

V~této úloze si vyzkoušíme sazbu titulní strany, matematických vzorců, prostředí a dalších textových struktur obvyklých pro technicky zaměřené texty\,--\,například Definice~\ref*{definice} nebo rovnice~\eqref{eq:3} na straně \pageref{eq:3}. Pro vytvoření těchto odkazů používáme kombinace příkazů \verb|\label|, \verb|\ref|, \verb|\eqref| a \verb|\pageref|. Před odkazy patří nezlomitelná mezera. Pro zvýrazňování textu jsou zde několikrát použity příkazy \verb|\verb| a \verb|\emph|.
\par Na titulní straně je použito prostředí \verb|titlepage| a sázení nadpisu podle optického středu s~využitím \emph{přesného} zlatého řezu. Tento postup byl probírán na přednášce. Dále jsou na titulní straně použity čtyři různé velikosti písma a mezi dvojicemi řádků textu je použito odřádkování se zadanou relativní velikostí 0,5\,em a 0,4\,em\footnote{Nezapomeňte použít správný typ mezery mezi číslem a jednotkou.}.

\section{Matematický text}

V~této sekci se podíváme na sázení matematických symbolů a výrazů v~plynulém textu pomocí prostředí \verb|math|.  
Definice a věty sázíme pomocí příkazu \verb|\newtheorem| s~využitím balíku \verb|amsthm|. Někdy je vhodné použít konstrukci \verb|${}$| nebo \verb|\mbox{}|, která říká, že (matematický) text nemá být zalomen. 

\begin{definice} \label{definice}
\textup{Zásobníkový automat} (ZA) je definován jako sedmice tvaru $A=(Q,\Sigma,\Gamma,\delta,q_0,Z_0,F)$, kde:
\begin{itemize}
    \item $Q$ je konečná množina \emph{vnitřních (řídicích) stavů},
    \item $\Sigma$ je konečná \emph{vstupní abeceda},
    \item $\Gamma$ je konečná \emph{zásobníková abeceda},
    \item $\delta$ je \emph{přechodová funkce $Q \times (\Sigma\cup\{\epsilon\})\times\Gamma\rightarrow 2^{Q\times\Gamma^\ast}$},
    \item $q_0 \in Q$ je \emph{počáteční stav}, $Z_0 \in \Gamma$ je \emph{startovací symbol zásobníku} a $F \subseteq Q$ je množina \emph{koncových stavů}.
\end{itemize}
\end{definice}

Nechť $P=(Q,\Sigma,\Gamma,\delta,q_0,Z_0,F)$ je ZA. \emph{Konfigurací} nazveme trojici $(q,w,\alpha) \in Q \times \Sigma^\ast \times \Gamma^\ast$, kde $q$ je aktuální stav vnitřního řízení, $w$ je dosud nezpracovaná část vstupního řetězce a $\alpha=Z_{i_1}Z_{i_2}\ldots Z_{i_k}$ je obsah zásobníku.

\subsection{Podsekce obsahující definici a větu}

\begin{definice} 
\textup{Řetězec $w$ nad abecedou $\Sigma$ je přijat ZA} $\mathnormal{A}$~jestliže $(q_0, w, Z_0)\overset{\ast}{\underset{A}{\vdash}}(q_F,\epsilon,\gamma)$ pro nějaké $\gamma\in\Gamma^*$ a $q_F \in F$. Množina $L(A)=\{w\ |\ w \mbox{ je přijat ZA } \mathnormal{A}\} \subseteq \Sigma^\ast $ je \textup{jazyk přijímaný ZA} $\mathnormal{A}$. 
\end{definice}

\begin{veta}
Třída jazyků, které jsou přijímány ZA, odpovídá \textup{bezkontextovým jazykům}.
\end{veta}

\section{Rovnice}
Složitější matematické formulace sázíme mimo plynulý text pomocí prostředí \verb|displaymath|. Lze umístit i několik výrazů na jeden řádek, ale pak je třeba tyto vhodně oddělit, například příkazem \verb|\quad|. 
\begin{displaymath}
    1^{2^3} \not= \Delta^1_{\Delta^2_{\Delta^3}} \quad y^{11}_{22}-\sqrt[9]{x+\sqrt[7]{y}} \quad x>y_1 \leq y^2
\end{displaymath}

\noindent V~rovnici~\eqref{eq:2} jsou využity tři typy závorek s~různou \emph{explicitně} definovanou velikostí. Také nepřehlédněte, že nasledující tři rovnice mají zarovnaná rovnítka, a použijte k~tomuto účelu vhodné prostředí.
\begin{eqnarray} 
    -\cos^2 \beta & = & \frac{\frac{\frac{1}{x}+\frac{1}{3}}{y}+1000}{\prod\limits _{j=2}^8 q_j} \label{eq:1} \\
    \biggl(\Bigl\{b \star \bigl[3 \div 4\bigr]\circ a\Bigr\}^\frac{2}{3}\biggr) & = & \log_{10}x \label{eq:2} \\
    \int_a^b f(x)\,\mathrm{d}x & = & \int_c^d f(y)\,\mathrm{d}y \label{eq:3}
\end{eqnarray}
\noindent V~této větě vidíme, jak vypadá implicitní vysázení limity $\lim_{m\rightarrow\infty}f(m)$ v~normálním odstavci textu. Podobně je to i s~dalšími symboly jako $\bigcup_{N\in\mathcal{M}}N$ či $\sum^m_{i=1}x^2_i$. S~vynucením méně úsporné sazby příkazem \verb|\limits| budou vzorce vysázeny v~podobě $\lim\limits _{m\rightarrow\infty}f(m)$ a $\sum\limits _{i=1}^m x^4_i$.

\section{Matice}

Pro sázení matic se velmi často používá prostředí \verb|array| a závorky (\verb|\left|, \verb|\right|).
$$
    \mathbf{B} = \left| 
    \begin{array}{cccc}
         b_{11} & b_{12} & \cdots & b_{1n} \\
         b_{21} & b_{22} & \cdots & b_{2n} \\
         \vdots & \vdots & \ddots & \vdots \\
         b_{m1} & b_{m2} & \cdots & b_{mn}
    \end{array}
    \right| = \left|
    \begin{array}{cc}
         t&u\\
         v&w
    \end{array}
    \right| = tw-uv
$$$$
    \mathbb{X}=\mathbf{Y}\Leftarrow\!\Rightarrow\left[
    \begin{array}{ccc}
        & \Omega+\Delta & \hat{\psi} \\
        \vec{\pi} & \omega & \\
    \end{array}
    \right]\not=42 
$$

Prostředí \verb|array| lze úspěšně využít i jinde, například na pravé straně následující rovnice. Kombinační číslo na levé straně vysázejte pomocí příkazu \verb|\binom|.
$$
    \binom{n}{k}=\left\{
    \begin{array}{cc}
         0 & \text{pro } k < 0 \hfill \\
         \frac{n!}{k!(n-k)!} & \text{pro } 0 \leq k \leq n \hfill \\
         0 & \text{pro } k > 0 \hfill
    \end{array}
    \right.
$$

\end{document}