\documentclass[a4paper, 11pt, hyphens]{article}
\usepackage[top=3cm, left=2cm, text={17cm, 24cm}]{geometry}
\usepackage[czech]{babel}
\usepackage[utf8]{inputenc}
\usepackage{times}
\usepackage[hidelinks, linkcolor=black, unicode, hyperfootnotes=false]{hyperref}

\providecommand{\uv}[1]{\quotedblbase#1\textquotedblleft}

\begin{document}

\thispagestyle{empty}
\begin{titlepage}
    \begin{center}
        \textsc{\Huge Vysoké učení technické v~Brně\\ \medskip
        \huge Fakulta informačních technologií}\\ \bigskip
        \vspace{\stretch{0.382}}
        {\LARGE Typografie a publikování\,--\,4.\ projekt\medskip \\} 
        {\Huge Bibliografické citace}
        \vspace{\stretch{0.618}}
    \end{center}
    {\Large \today \hfill Marek Gergel \\}
\end{titlepage}

\section{Úvod do typografie}

V~typografii sa používajú rôzne typy písma, ktoré sa využívajú na rôzne účely.
Ich výber je dôležitý, pretože písmo môže ovplyvniť čitateľa a jeho vnímanie textu\cite{NYTimes2020}.

\subsection{Písmo}

Písmo je vizuálna reprezentácia abecedy a iných symbolov, ktoré sú používané na písanie textu. Písmo sa skladá z~rôznych prvkov, ako sú napríklad písmená, číslice, interpunkčné znamienka a iné symboly, ktoré sa môžu líšiť svojím tvarom, veľkosťou, hrúbkou, sklonom a ďalšími vlastnosťami\cite{Bringhurts1992}.

\subsection{Patkové písmo}

Patkové písmo alebo serifové písmo je písmo, ktoré má na konci písmen a číslic zakončenia pomocou ornamentálnej čiarky, tzv. patky.
Tá sa využíva na zlepšenie čitateľnosti textu, práve vďaka optickému zarovnaniu. Čitatelnosť samotných písmen môže byť ale zhoršená\cite{Hlavenka1995}\cite{Uhlirova2016}.

\section{Times New Roman}

Times New Roman je patkové písmo, ktoré navrhol v~roku 1931 typograf Stanley Morison pre denník The Times. Prvý krát sa použilo 3.\ októbra 1932\cite{NYLib}.
Písmo patrí medzi najpopulárnejšie písma na svete a je používané v~mnohých publikáciách, napríklad v~knižnom formáte, časopisoch, novinách, ale aj na internete\cite{TimesWiki}.

\subsection*{Použitie v~LaTeXe}

Použitie písma Times New Roman v~LaTeXe je veľmi jednoduché. Stačí zadať príkaz \texttt{\textbackslash usepackage{times}}.
Písmo je možné použiť v~rôznych veľkostiach, napríklad \texttt{\textbackslash large} alebo \texttt{\textbackslash huge}, ale aj v~rôznych štýloch, napríklad \texttt{\textbackslash textsc{Times New Roman}} alebo \texttt{\textbackslash textit{Times New Roman}}\cite{Rybicka2003}.

\newpage
\bibliographystyle{czechiso}
\renewcommand{\refname}{Literatura}
\bibliography{proj4}

\end{document}